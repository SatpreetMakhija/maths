\documentclass{report}
\usepackage{amsthm, amssymb, amsmath}
\usepackage{todonotes}
\usepackage[colorlinks=true, linkcolor=blue]{hyperref} % Enable hyperlinks
\begin{document}
\listoftodos

\chapter{Introduction}
\section{What is analysis?}
Real analysis is the analysis of real numbers, sequences and series of real numbers and real-valued functions (functions which have range as real numbers). 
\section{Examples}
The examples in this section will show why understanding of real numbers is important. We'll see situations where if we don't really understand what these real numbers are, we won't be to make correct decisions. 
\subsection{division by zero}
$ac = bc \implies a = b$. But, this does not work when $c= 0$. What is this example telling us? Whenever we are cancelling like how we did above, we are ruling out that $c =0 $ for all practical purposes.Because cancellation here actually means division by $c$, we must make sure $c \neq 0$. 
\subsection{divergent series}
Take $S = 1 + \frac{1}{2} + \frac{1}{4} + + \frac{1}{8} + \dots $, what is $S$?. We can use this trick: Multiply both sides by $2$. We get $2S = 2 + 1 + \frac{1}{2} + \frac{1}{4} + \dots$. 
\[
2S = 2 + S \implies S = 2 
.\] 
If we have another sum $S = 1 + 2 + 4 + 8 + 16 + \dots$, using the same trick leads to  \[
2S = 2 + 4 + 8 + 16 + \dots\implies 2S + 1 = S \implies S = -1
.\] 
Clearly, that should not be the case. \todo{EOB need answer} My initial guess is that we cannot apply this trick for a divergent series but can for a convergent series. Why, I need to find out.


\subsection{Divergent sequence}
Let $x$ be a real number and let $L$ define the limit as follows:
\[
L = \lim_{n\to \infty} x^{n} 
.\] 
Let $n = m + 1$. 
\[
\therefore L = \lim_{m+1 \to \infty} x^{m+1}  
.\] 
\[
L = x \lim_{m+1 \to \infty} x^{m}
.\]
\todo{Is it allowed to do take $x$ out of the limit?}
\[
\because m+1 \implies \infty, m \implies \infty 
.\] 
\[
L = x \lim_{m \to \infty} x^{m}
.\] 
\[
L = xL 
.\] 
Either $x = 1$ or $L = 0$. 
This means $L = \lim_{n \to \infty} x^{n} = 0, x \neq 1$. But, this does not make sense because clearly when instantiated for $x = 2$, $L = \lim_{n \to \infty} 2^{n} \neq 0$ 
My guess is that the part where we move $x$ out of the limit so easily should not be allowed.


\subsection{}
\subsection{}
\subsection{Interchanging of integrals}
We sometimes use this trick when integrating $\int \int f(x,y) dx dy = \int \int f(x,y) dy dx$. But this too can lead to issues sometimes. Look at the example in the book for specific example. \todo{EOB need answer on interchanging integrals}

\[
	\int_{0}^{\infty} \int_{0}^{1} \left( e^{-xy}-xye^{-xy}dy dx \right) = \int_{0}^{1} \int_{0}^{\infty} \left( e^{-xy}-xye^{-xy}dxdy \right)      
.\] 
We'll see in the book that the results of the two are different (LHS and RHS is not the same).

\subsection{Interchanging limits}
\subsection{}
\subsection{}


\chapter{Natural numbers}
Just to reiterate the different kinds of numbers we'll deal with in this book, here are they in increasing order of sophistication. Natural numbers, $\mathbb{N}$,  integers, $\mathbb{Z}$, rational numbers, $\mathbb{Q}$, real numbers, $\mathbb{R}$ and complex numbers, $\mathbb{C}$. 

\section*{Axiom and definition counts in the chapter}
\paragraph{Commentary: }Since axioms and definitions are the most important part of a book, we'll keep a track of the number of axioms and definitions in each chapter. 
5 axioms for Peano's axioms to define natural numbers,
1 definition of addition,
, 1 definition of positive numbers, 1 defintion of ordering (greater than). 

\section{The Peano axioms}
The Peano axioms is one of the ways of defining natural numbers. There are other approaches such as using sets.
We define natural numbers as that set of elements for which the following axioms hold:
\subsection{0 is a natural number.}
\subsection{If $n$ is a natural number, then $n++$ is also a natural number}
Note, we assume that there exists an operation with symbol $++$ that means increment. This axiom allows us to move forward in our count.
\subsection{$0$ is not the successor for any natural number, i.e., $n++ \neq 0 \forall n$}
This ensures we don't circle back to $0$. Imagine if we defined  $\left( 0++ \right) ++ $ as $0$. Axioms $2.1.2$ and $2.1.1$ would still hold but we know we don't want that in our natural number system. This axiom ensures that. 
\subsection{$n \neq m \implies n++ \neq m++$}
This axiom ensures that we don't reach an upper bound. If we define $2++$ and  $3++$ both as $3$, none of the above axioms are violated and yet we know we don't want an upper bound to our natural number system. Take the next number, $(3++)++ = 3++ = 3$, we have reached an upper bound for us.
Another way to state this axiom is in its contrapositive form. $n++ = m++ \implies n = m$.
\subsection{Principle of Mathematical Induction}
It might be surprising right now why we are including this as an axiom because it is part of the larger logic scheme. Why do we need to explicitly say this. My guess is that it is needed because there are other systems where this principle does not hold. \todo{find why math induction principle is needed as a Peano axiom to define natural numbers}
The principle is if some property $P$ holds for $0$, that is, if $\left( P\left( 0 \right)  \right) $ is true and  $P\left( n \right) \implies P\left( n+1 \right) $ is true, then property $P$ is true for all elements in the number system. 

This axiom allows us to define properties of objects for all elements.
\paragraph{Commentary: } \todo{Why do we fix other variables and induct on only one?}

\paragraph{Note our definition of natural numbers is axiomatic and not \textit{constructive}.} What I mean by that is we have not defined \textit{what} a natural number is but rather properties that a natural number holds. Any number system that follows the above mentioned axioms is a natural number system.

\subsection{Recursive definition of sequence: Proposition} Suppose for each natural number $n$, we have some function $f_n: \mathbb{N} \mapsto \mathbb{N}$. Let $c$ be a natural number. Then, we can assign a unique natural number $a_n$ to each natural number $n$ such that $a_0 = c$ and $a_\left\{ n++ \right\} = f_n\left( a_n \right) $ for each natural number $n$.
\paragraph{commentary} The goal is to show that each value of the sequence $a_i$ is defined only once and because we are looking at a recursive defintion we do not revisit $a_i$ again when going forward. 
\paragraph{proof} We'll prove this by induction. Take the base case, $a_0 = c$. Because of axiom $2.1.3$, going forward, any $a_{m++} $ will not redefine $a_0$. For example, if axiom $2.1.3$ was actually violated and say $3++ = 0$. Then, $a_{3++} = f_3\left( a_3 \right) = a_0$. Note, we redefined $a_0$ without this axiom.

Let's assume that $a_n$ by the given recursive formula uniquely defined $a_n$. We need to this also holds for $a_\left( n+1 \right) $. Any natural number after $n++$, say we call it $m++ $ will redefine $n++$ because of axiom $2.1.4$. Note, this axiom was introduced to ensure the successors are always bigger and that's what this axiom ensures. Because we cannot circle back, the previously defined $a_i$ remains as is.

\section{Addition}
\subsection{Definition: Addition of natural numbers}
Let $m$ be a natural number, we define addition of zero to $m$ as $0 + m := m$. We give a recursive defintion for adding. Let's say we know how to perform $n + m$. Therefore, adding  $n++$ to $m$ is defined as $\left( n++ \right) +m := \left( n+m \right) ++$.

For example, say, we need to add $2 + 5$.  $\left( 1++ \right) +5 = \left( 1 + 5 \right) ++ = \left( \left( 0++ \right) +5 \right) ++ = \left( \left( 0 + 5 \right) ++ \right) ++ = \left( 5++ \right) ++ = 6++ = 7 $
\paragraph{Commentary:} This definition is enough to deduce everything about addition of natural numbers!
\subsection{Lemma} For any natural number $n$, $n+0 = n$. 


\paragraph{Commentary:} Note, in our recursive defintion, we have defined only $0+n=n$ and we don't know yet that for any $a,b \in \mathbb{N}$, $a+b=b+a$.
\paragraph{Proof:} Take base case, $n = 0$. $0+0=0$. This is true by our recursive defition's base case. Assume, $n+0 = n$ for $n$. We need to show $(n++)+0 = n++$ and we are done with our proof by induction. 

 $\left( n++ \right) +0 = \left( n + 0 \right) ++ $ by our defintion of addition. $\left( n+0 \right) ++ = n++$ because of our induction assumption for $n$. Hence, proved.

 \subsection{Lemma} For any natural numbers $n,m$ $n+\left( m++ \right) = \left( n+m \right) ++$.

\paragraph{Commentary:}Again, we cannot deduce this from our recursive defintion  $\left( n++ \right) + m = \left( n+m \right) ++$ because we do not know $a+b = b+a$. 

\paragraph{Proof: } We'll try this by induction on $n$ (and keep $m$ fixed). Let's take the base case at $n=0$. $0+\left( m++ \right) = m++ = \left( 0 + m \right) ++ $. The stamenent holds for the base case.

Assume, $n+\left( m++ \right) = \left( n+m \right) ++$. We now need to show that $\left( n++ \right) + \left( m++ \right) = \left( \left( n++ \right) + m \right) ++ $

$\left( n++ \right) + \left( m++ \right) = \left( n+\left( m++ \right)  \right) ++ $ by the recursive defintion of addition.
$\left( n + \left( m++ \right)  \right) ++ = \left( \left( n+m \right) ++ \right) ++ $ by the the induction step taken for $n$. 
$\left( \left( n++ \right) +m \right) ++$ by the defintion of recursive defntion of addition. Hence, proved. 

\subsection*{Corollary}
$n++ = n + 1 $ 
\paragraph{Commentary: }Why would this hold because of Lemma  $2.2.2$ and Lemma $2.2.3$. 

\paragraph{Proof:} $n+1 = n + \left( 0++ \right) = \left( n+0 \right) ++$ by Lemma $2.2.3$. $\left( n+0 \right) ++ = n++$ because of Lemma $2.2.2$. Hence, proved.

\subsection{Proposition} Addition is commutative. For any natural numbers $n,m$, $n+m = m+n$. 

\paragraph{Proof: } We can prove this by induction. (note, anytime there's a proof of applying it for all numbers, induction really helps). We'll fix $m$. Take  $n=0$. $0+m = m$ by defition of base case for addition. $m+0 = m$ by Lemma $2.2.2$. Base case is done.

Assume $n+m = m+n$. This is our induction step. We now need to show it holds for $n++$, i.e., $\left( n++ \right) + m = m + \left( n++ \right) $. 
LHS, $\left( n++ \right) + m = \left( n+m \right) ++	$ by the recursive definition of addition. $\left( n+m \right) ++ = \left( m+n \right) ++$ by our induction step.
RHS, $m+ \left( n++ \right) = \left( m+n \right)++ $ by Lemma $2.2.3$.

LHS = RHS. Hence proved. $\qed$

\subsection{Proposition} Addition is associative. For any natural numbers $a,b,c$, we have $(a+b)+c = a+\left( b+c \right) $.
\todo{Write proof for Addition is associative}

\subsection{Proposition} Cancellation law. Let  $a,b,c$ be natural numbers such that $a+b = a + c$. Then we have $b=c$.
 \paragraph{Commentary: }Note, we do not have the concept of negative numbers or subtraction yet.
\paragraph{Proof: }We'll prove this by induction. Fix $b,c$. Take base case,  $a = 0 $. $0+b = 0 + c \implies b = c$ because of base case of definition of addition.

Assume $a+b = a+c \implies b = c$. We need to show $\left( a++ \right) +b =  \left( a ++ \right) + c \implies b = c $.

\[
\left( a++ \right) + b = \left( a++ \right) +c 
.\] 
 \[
	 \left( a+b \right) ++ = \left( a+c \right) ++ \tag{by defintion of addition}
.\]
\[
	b++ = c++ \tag{by assumption of the induction step}
.\] 
\[
b = c
.\] 
Hence, proved by induction.  $\qed$

\subsection{Definition} Positive natural numbers. A natural number $n$ is said to be positive iff it is not equal to $0$. 

\subsection{Proposition} If $a$ is positive and $b$ is a natural number, then $a+b$ is positive.
 \paragraph{Proof: } We'll use induction on $b$. Let $b = 0$. $a + b = a + 0 =  a$. Base case holds. Assume $a+b$ is positive. Then we need to show $a+\left( b++ \right) $ is positive. $a + \left( b++ \right) = \left( a+b \right) ++$ by Lemma 2.2.3. $\left( a+b \right) ++ $ cannot be zero because of Axiom $2.3$ ($n++ \neq 0$ ). Hence, $\left( a+b \right) ++$ is positive. This closes the induction. $\qed$

 \subsection{Corollary} If $a$ and $b$ are natural numbers such that $a + b = 0$ then $a=b=0$.
\paragraph{Proof: } We'll prove this by contradiction. Assume $a+b = 0 $ but $a \neq 0$ or $b \neq 0$. In either case, $a+b=0$ where one of them is not zero. But, by Proposition $2.2.8$ that cannot be the case as $a+b$ must be positive (not zero). Hence, contradiction and $a = b= 0$.  $\qed$


\subsection{Lemma} Let  $a$ be a positive number. Then there exists exactly one natural number $b$ such that $b++ = a$.
\paragraph{Commentary: } Note, this means there is exactly one element behind $a$. There seems to be an order between the natural numbers.

\subsection{Definition} Ordering of natural numbers. Let $n$ and $m$ be natural numbers. We say $n$ is \italics{greater than or equal to $m$} iff we have  $n = m + a$ for some natural number $a$. We write \italics{greater than or equal to} as $n\geq m $ or  $m \leq n$. We say that $n$ is strictly greater than $m$ iff $n \geq m$ and $n \neq m$. 

\paragraph{Commentary: }Note, how we have used the concept of some $a$ existing such that adding it to $m$ gives us $n$. This also helps us show why there is no largest natural number because for every $n$, $n++ > n$. Hence, for the number $n ++$, there's another larger number $\left( n++ \right) ++$. 


\subsection{Proposition} Basic properties of order for natural numbers. Let $a,b,c \in \mathbb{N}$. Then
\begin{enumerate}
	\item (Order is reflexive) $a \geq a$
	\item (Order is transitive) If  $a \geq b$ and $b \geq c$, then $a \geq c$
	\item (Order is anti-symmetric) If  $a \geq b$ and $b \geq$, then $a =b$.
	\item (Addition preserves order)  $a \geq b$ iff $a+c \geq b+c$
	\item  $a < b$ iff $a++ \leq b$.
	\item  $a < b $ iff $b = a + d$ for some positive number $d$. 
\end{enumerate}
\paragraph{Proof: }
\begin{enumerate}
	\item 1
		$a \ge a$ iff there exists  $n$ such that $a = a + n$. Let $n=0$. Therefore, $a= a + 0 = a$. Hence, proved.
	\item
		It is given that  $a \ge b$ and $b \ge c$. We need to show this implies $a \ge c$. There must bexist $m$ such that  $a = b + m$ and  $n$ such that $b = c + n$. Substituting value of $b$, $a = c + n + m \implies a = k + c$ where  $k$ is some natural number. Hence, by definition of ordering of natural numbers, $a \ge c$.  $\qed$   
	\item
		 $a = b + m$ for some  $m$ and $b = a + k$ for some $k$. Substitute value of $b$, $a = a + k + m \implies a + 0 = a + (k+m) \implies 0 = k +m $ by cancellation law (Proposition $2.2.6$). The only way this holds is iff $k = m = 0$.
	This is because there are four situations $k,m \ne 0 \lor k=m=0, \lor k = 0 \land m \ne 0 \lor k \ne 0 \land m=0$. In all situations except $k=m=0$, we'll have a situation where  Axiom  $2.1.3$ is violated because we'll have an equation of form  $n++=0$. Hence, $a = b + m = 0 = b \implies a = b$. $\qed$ 
	\item 
	Assume $a \ge b$. This means $a = b + m$ for some $m$. Adding $c$ on both sides,   (for which we'll need to prove this logic TODO), $a + c = b + m + c \implies \left( a +c \right) = m + (b+c)$. (We have only rearranged the terms using Proposition $2.2.4$). Therefore, by definition of greater than, we have  $a + c \ge b+c$.
	Now assume  $a +c \ge b + c $. We need to show  $a \ge b$.  $a +c= b +c + d$ for some  $d$. 
By cancellation law, $a = b + d$. Hence, $a \ge b$.  We are done with our proof now.
	\item Assume $a < b$. We need to show $a++ \le b$. Or in other words  $a+c=b$ for some $c$ and $a \ne b$. We'll prove this by contradiction. Assume $\neg \left( a + c \ne b \right) \land \left( a \ne b  \right) = a+c \ne b \lor a = b$. We'll show that in both cases of or, we have a contradiction. $a ++ \le b \implies \left( a++ \right) +m = b = \left( a+m \right) ++ = b$. Since $a=b$ (by our assumption, this is the second case of or), $\left( b+m \right) ++ = b$. In both cases, when $m = 0 \lor m \ne 0$, we have a contradiction as $b++ \ne b$ as this would imply $0++ = 0$. (This is a contradiction of Axiom 2.1.3). Now, we'll show the other case of or also leads to contradiction, i.e., $a+c \ne b$ will lead to contradiction. Since $a++ \le b \implies \left( a++ \right) +c = b \implies \left( a+c \right) ++ = b \implies \left( c++ \right) +a = b$. But there does not exist a number  $b$ such that  $a+c = b$ (Note, here there is no difference between  $c$ and  $c++$ for us because all we care about is some number in \mathbb{N}. Hence, contradiction. Hence, our initial assumption $\neg \left( a+c \ne b \right) \land \left( a \ne b \right) $ must be wrong. Therefore, $a+c = b \and a \ne b$. Hence, we are done with left-to-right side for our proof. We now need to show right to left follows. Assume $a++ \le b$. We need to show $a < b$. We'll again show this by contradiction. Assume $\neg \left( a < b \right) $$a++ \le b \implies \left( a++ \right) +c = b $ for some $c$.  $a++ = a + 1 = a + (0++) = a++$. Therefore, we'll replace $a++$ in our equation. We get $a+1+c = b \implies a+d = b$ where $d = c+1$. But, we now have a condition that implies $a < b$ by the definition of greator than. Hence contradiction. Hence, $a < c$. Now we are done with both directions of our proof. $\qed$
 

\end{enumerate}

\subsection{Proposition: } Trichotomy of order of natural numbers. Let $a$ and $b$ be natural numbers. Then exactly one of the following statements is true: $a < b, a = b, a > b$.

\paragraph{Commentary: } This is a great example to show how we can pin down what we want to say in exact terms using mathematics. We need to show that exactly one of three statements is true. We first show that no two statements can be true together. We also show that the three together also does not hold true. Then we show at least one of them is true. In simple terms, we show one statement will definitely be true and two statements cannot be true together. Hence, only one statement will be true. 

\paragraph{Proof: } We show first no more than one statement is true. If $a < b$, by definition of greater than $a \ne b$. If $a > b$, by definition of smaller than $a \ne b$. If $a < b$ and $a > b$, then by Proposition $2.2.12(3)$, $a=b$ which is a contradiction. Hence, no more than one statement can be true.

Now, we show at least one of the statements is true. We'll use induction. Fix $b$ and induct on $a$. Base case, $a=0$. $0 \le b$ for all $b$. This is because  $b=a \lor b \ne a$. If $b=a$, $0 \le b$ holds. If $b \ne a$, and we know $a = 0$, therefore, $b$ must be some positive number and hence $b > 0$. Therefore, we show that in both cases, $0 \le b$. Now suppose we have proven the proposition for $a$ and now we want to prove the proposition for $a++$. There are three cases: $a < b, a=b, a > b$. If $a>b$, by proposition $2.2.12(e)$, we get $a++ \le b$. This case holds. (Note, we show that $a++ = b \lor a++ < b$. I know we want to show only one statement holds overall but this part of the proof is about at least one of the statemnets is true, if both are true we don't care). If $a = b$. $a=b \implies a++ = b+1$. By defintion of greater than, $a++ \ge b$. Therefore, $a++ > b$ or  $a++ = b$. If  $a++ = b \implies a++ = a $ which cannot be the case. Therefore,  $a++ > b$. We are done with this case as well. If  $a < b$, then by Proposition $2.2.12(5)$, $a++ \le b$ and in both cases we are done with our induction. Hence proved.   

\subsection{Proposition: Strong principle of induction}
Suppose there exists a natural number $m_0$. Suppose we also have a property $P(m)$ for some natural number $m$ such that the following as true: If $P(m^{`})$ is true for all $m_0 \le m^{`} < m$, then $P(m)$ is true. Then we conclude $P$ is true for all $m \ge m_0$.

\paragraph{Commentary}: Is strong principle of induction stronger than our Principle of Induction Axiom (Proposition 2.1.5)itself? If strong principle of induction is proved using Principle of Induction Axiom then it's not really a stronger claim than Principle of Induction Axiom. Then, why call it strong?

The proposition hinges on the if then implication. It says that if proposition holds till $m$, it holds for $m++$. Note, this part is given, i.e., the implication itself is given as true. Then, we can conclude that $P$ is true for all $m$.


There are no numbers before $m_0$. And we are given that if  $P(m`)$ is true for all numbers before $m$, then $P(m)$ is true. So,  $P(m_0)$ is true. 

Let's talk about $P(m_1)$ then. It is given to us that if  $P(m`)$ is true for all numbers before  $m_1$, it implies $P(m_1$ is true. We know $P(m_0)$ is true. Hence, $P(m_1)$ is true.

\textbf{This process of induction does feel like a trick but it hinges on a given if-then statement where it states that $P$ is true for all numbers before $m$, $P$ is true for $m$. Again, this is a given to us.}


\section*{Exercises}
\subsection*{2.2.1}
 \paragraph{Q: } Prove Proposition $2.2.5$, i.e., $\left( a+b \right) +c = a + \left( b+c \right) $.

\paragraph{Proof: } We'll prove this by induction. Fix $b,c$. Take $a=0$ for base case. $\left( 0+b \right) +c = b +c $. This was LHS. RHS side, $0+(b+c) = b+c$. Base case holds. We'll take the induction step. Assume the given property holds for $a$. We need to show  $\left( \left( a++ \right) +b \right) +c = \left( a++ \right) + \left( b+c \right) $.
Take LHS, $\left( \left( a++ \right) +b \right) +c  = (\left( a+b \right) ++)+c = \left( (a+b)+c \right)++$. We only used the definition of addition for this manipulation. Take RHS, we'll get $\left( a+(b+c) \right)++ $ after similar manipulation. By the inductive step we know $a+(b+c) = (a+b)+c = k$ for some  $k$. LHS and RHS are equal. $k++ = k++$. Hence, induction closes here.

\subsection*{2.2.2}
\paragraph{Q:} Prove that if $a$ is a positive number then there exists exactly one number $b$ such that $b++ = a$. 
\paragraph{Commentary: } I have a feeling for contradiction here. Clearly, I have to show it in two parts, first that there exists such an number $b$ such that $b=a++$ and that $b$ is unique. For the first part, we don't need a contradiction. 
Hint says \textit{use induction}, so we'll try induction. 
\paragraph{Proof: } Base case, take $a=1$. Clearly, there exists  $b = 0$ since $0++=1$. There isn't any other number so we are good with the base case. Induction assumption step. Say, there exists exactly one  number  $b$ for  $a$. We need to show there exists a number $c$ for  $a++$.  $a++= \left( b++ \right) ++ $. Here $b++ =c $. Since, we know there exists only one  $b$,  $b++$ is also unique. \todo{not satisfied with proof, try again}

\subsection*{2.2.4}
why(1): $b$ is a natural number. It either is $0$ or a positive number. 

why(2): We need to show  $a > b \implies a++ > b$.  $a>b$, by defintion means  there exists  $n$ such that $a=b+n$. Adding  $1$ on both sides, $a++= b+\left( n++ \right) $. Clearly, by definition of greater than we can say  $a++ > b$.

why(3): We need to show  $a=b \implies a++ > b$.  $a++=b++ \implies a++ = b+1$. Clearly, by defintion of greater than $a++ > b$. $\qed$  



\subsection*{2.2.5}
\paragraph{Q: Prove Proposition 2.2.14 Strong principle of induction}
\paragraph{Proof:}
$(P\left( m_0 \right)$ is true. ($P(m)$ is true for all  $m \ge m_i \ge m_0$ implies $P(m++)$ is true)) $\implies$  $P(m)$ is true $\forall m \in \mathbb{N}$.
\paragraph{Commetary: } We'll use induction for this proof. The base case will be that $P(0)$ is true (we need to show that) and also as part of the base case, we need to show that $P(0)$ is true $\implies$, $P(0++)$ is true and that this implies $P(m)$ is true for all $m \in \mathbb{N}$. Then, as part of the induction step, we need to show if we assume  $A,B \implies C$ to be true for  $m$, then  $A,B \implies C$ is also true. Note, here $A$ is  $P(m_0)$ is true.  $B$ is $P(m_i) \implies P(m++) \forall m \ge m_i \ge m_0$ and $C$ is  $P(m) \forall m \in \mathbb{N}$ is true. Note, as part of the induction step as for the base case, we will assume that the first part of $B$ is true and show that the second part, i.e., the implication follows from it.

Take the base case $m_0 = 0$. It is given to use that $A, B$ is true and we need to show  $C$ follows from them. $B$ is true, therefore the implication of it is true, i.e.,  $P(0++)$ is true. From this, we can say that $P(1)$ is true. So, again we have $P(0)$ is true and by $B$ we will get $P(1++)$ is true. This way, we can show $P(m)$ is true $\forall m \in \mathbb{N}$. We are done with the base case.


Now, assume  $A,B \implies C$ is true for  $m$. We need to show  $A,B \implies C$ is also true. What do we get when we assume $A, B \implies C$ is true? Well, the exact statement that we need to prove. No point in repeating it here. What we need to show is that given  $P(m_0)$ is true, and $P(m++) \implies P((m++)++)$ is true, tells us that $P(m++)$ is true $\forall $m++$ \in \mathbb{N}$.
Any $m \in \mathbb{N}$  is written as $(((0++)++)++)...$ where  $++$ operation is repeated  $m$ number of times. We are given $P(1) \implies P(2)$ and so on and so forth. So, $P(0) \implies P(1)$ and by using the same logic we'll show that $P(1) \implies P(2)$. In this way, since every positive number is some  $m++$, we have shown that $C$ is true  $\forall m++ \in \mathbb{N}$. Hence, induction closed. \todo{very bad proof, not satisfied, try again.}

\subsection*{2.2.6}
\paragraph{Given: } $P(m++) \implies P(m)$.  $P(n)$ is true.
\paragraph{Show: } $P(m)$ is true $\forall m \le n$.
\paragraph{Proof:} Assume base case is about $n=0$. We are given $P(0++) \implies P(0)$ is true. $P(m)$ is true trivially as there is  $m \neg \le0$ Base case holds. Assume,  $A \implies B, C$ is true and together they imply  $C$ is true. We need to show the same for $m+++$. Here $n=m++$. And as part of given we have  $P((n++)++) \implies P(n++)$, we need to show that  $P(m)$ is true $\for m \le n++$. Now, because of the given, $P(m)$ is true  $\forall m \le n$. We just need to show it also holds for  $n=m++$, But that is also given as part of the assumption that  $P(n++)$ is true. Hence, we are done with the induction. $\qed$. \todo{bad proof, not convinced}







\chapter{Set Theory}
\section*{Definition and Axiom count}
There are 11 axioms used to construct set theory. These 11 axioms are called \textit{Zernelo-Frankel} axioms. There is a $12$th axiom called the \textit{Axiom of Choice}  that we don't need now but will need at some point of time. \todo{Why is Axiom of Choice needed/not needed now?} There are 13 definitions in this chapter. Note, these also include the defintion of functions, types of functions as well because the axioms make use of them in their statements.
\todo{Why are definitions secondary to axioms?} While it is true that we can have infinite definitions and hence makes it redundant to keep a track of them but without defintions, 
\paragraph{Commentary: }What does obeying the axiom of substitution mean? Take example, for two natural numbers, $a,b$ we have $a=b$. Then, any other equation where $a$ is referred, I can substitute $b$ for it. Similarly, even for sets, we have the concept of equality. For any two sets,  $A,B$ we have defined equality, $A=B$. Therefore, we can write $x \in A$ as  $x \in B$. \todo{Find if axiom of substitution is defined for operation or elements}

\section{Axiom}
If  $A$ is a set, then $A$ is also an object. In other words, given two sets $A,B$ it is meaningful to ask if $A \in B$ or $B \in A$.

\section{Axiom}
Empty set 

\section{Axiom}
Singelton and pair sets
\paragraph{Commentary: } This axiom is about for a given object $a$, there exists a set  $S$ such that  $S={a}$, i.e.,  $a$ is the only element of  $S$. Another way of stating the same thing is by saying that if  $y \in S \implies y = a$. Similary, we have a pair set property for a set. Given two elements  $a,b$, there exists a set  $S$ whose only elements are  $a,b$.

There is only one singleton set for each object $a$ by defintion $3.1.4$ (equality definition). Say, there were two singleton sets for $a$. Every element  $y$ of  $S_1$ would belong to  $S_2$ because  $y=a$ and by definition of  $S_2$ being a singleton set, each  $y \in S_2 \iff y=a$. Similarly,  every element of  $S_2$ belongs to  $S_1$. Hence,  $S_1 = S_2$.  

\section{Axiom: Union of sets}
Given two sets  $A,B$, there exists a set called /\textit{union} and written as  $A \cup B$ which elements belong to  $A$ or  $B$ or both. \todo{In set theory, why is union an axiom but intersection a definition?}

\subsection*{3.1.12 Remark}
If $A,B,A_1$ are three sets and $A=A_1$, then $A \cup B = A_1 \cup B$. By definition of equality, any $x \in A \implies x \in A_1$ and vice versa. Any element  $x \in A \cup B$ implies  $x \in A \lor x \in B$ by defintion of pairwise union. Since,  $x \in A \implies x \in A_1$, we will replace the belongs to part. We can do this because by the defintion of equality, $\in$ allows axiom of substitution. Therefore,  $x \in A \cup B \implies x \in A_1 \cup B$ and vice versa. Hence, $A \cup B = A_1 \cup B$.  

\paragraph{Commentary: } This axiom allows us to create larger sets, i.e., sets with more than two elements. 

\subsection*{3.1.13 Lemma} $\left( A \cup B \right) \cup C = A \cup \left( B \cup C \right) $ (associativity)

\paragraph{Commentary: } To prove this we have a statement which needs us to show all elements on the side of the set $X$ also belong to the set $Y$ on the right hand side and vice versa.

\paragraph{Proof: } Let $x \in \left( A \cup B \right) \cup C$. We need to show $x \in A \cup \left( B \cup C \right) $. First, say $x \in A \cup B$. $x \in A \cup \left( B \cup C \right) = x \in A \lor x \in B \cup C$. If $x \in A \implies x \in A \lor x \in B \cup C$. If $x \in B$, we need to show  $x \in A \lor x \in B \cup C $ holds. If  $x \in B \implies x \in B \cup C$. (why?) because  $x \in B \cup C = x \in B \lor x \in C$ and  $x \in B$ holds.

Now, we need to show if $x \in C \implies x \in A \cup \left( B \cup C \right) $. Again $x \in C \implies x \in B \cup C$. We are done with this second part as well. 

Similarly, we can show from LHS to RHS as well. Hence proved.  $\qed$

\subsection*{3.1.18 Proposition} Sets are partially ordered by set inclusion. If  $A \subseteq B$, and  $B \subseteq C$, then  $A \subseteq C$. 

\section{Axiom: Axiom of specification}
\paragraph{Commentary}: This axiom allows us to create subsets from larger sets.

Given a set $S$ and a property  $P$, there exists a subset of  $S$, such that elements of  $x \in S$, for which  $P(x)$ is true, belong to this subset.

$y \in \left\{ x \in A : P(x) \right\} \iff y \in A \land P(y) $. This is how we mathematically we capture the above statement. 

\subsection*{3.1.23 Definition (Intersection)}
The \textit{intersection} of two sets written as $S_1 \cap S_2$ is defined to be the set $S_1 \cap S_2 := \left\{ x \in S_1 : x \in S_2 \right\} $. Note, how beautifully we used the axiom of specification which is a more general form of axiom where we talk about a generic property of the element and separate the set according to that property. Belongingness is also a property. This axiom, thus, gives us the ability to define the concept of intersection.


\section*{3.1.28 Sets form a boolean algebra}
\paragraph{Commentary: } \todo{Sets form a boolean algebra means what?} Is it the case that there are other objects that form a boolean algebra. What properties are quired for an object to satisfy boolean algebra? What lies at the heart of boolean algebra?

\todo{Why does the complementation relation $A \implies X/\ A$ create this duality in Morgan's Laws, i.e., unions convert into intersectin and vice versa?}

\section{Axiom of replacement}
\paragraph{Commentary: } This axiom helps us in converting one element into another of a set. Say, we have a set $S = {1,2,3}$ and we want to convert it to  $S = {4,5,6}$, none of the existing axioms will allow us to do that. Why are existing axioms not enough?


Suppose that for any object $x \in S$ and  $y$, there exists a property  $P(x,y)$ such that for every  $x$, there exists at most one  $y$ such that  $P(x,y)$ is true, then, there exists a set $\left\{ y:P(x,y) for some x \in S\right\} $ such that for any object $z$ in this set we have  $z \in \left\{ y:P(x,y) for some x \in S \right\} \iff P(x,z) is true for some x \in S $. 

\paragraph*{How will you combine axiom of separation with axiom of replacement?} Let's define a set using axiom of separation $\left\{ x:P(x) \right\} $. Now, how to apply axiom of replacement on this. Assume $\left\{ f(x): x \in A; P(x) \right\} $. 

\section{Axiom: Infinity}
\paragraph{Commentary: }This axiom says that there exists a set $N$ whose elements satisfy the property of Peano's axioms. Hence,  this axiom helps us show that $N$ is a set.

\section*{Exercises}
\subsection*{3.1.1} Reflexive, symmetric and transitive is followed by the defintion of equality. Reflexive here means $A = A$, symmetric means  $A_1 = A_2 \implies A_2 = A_1$ and transitive means $A_1 = A_2 \land A_2 = A_3 \implies A_1 = A_3$.
Reflexive is simple because each  $x \in A \implies x \in A$. Symmetric because the if part gives us $A_1 = A_2$. And by the defintion of equality, $A_2 = A_1$ because any element $x \in A_2 \implies x in A_1$ by defintion of equality of the given statement. Transitivity. Let $A_1 = A_2$ and $A_2 = A_3$. We need to show $A_1= A_3$. Let  $x \in A_1$. Since $A_1 = A_2$, $x \in A_2$ and similarly $x \in A_3$. Going in reverse direction also gives us the same situation, i.e.  $x \in A_3 \implies x \in A_1$. Hence, by defintion of equality $A_1 = A_3$. $\qed$
 
\subsection*{3.1.2}
eq, 1,2,3 prove sets  $\phi, \left\{ \phi \right\} , \left\{ \left\{ \phi \right\}  \right\}, \left\{ \phi, \left\{ \phi \right\}  \right\}  $ are distinct, i.e., none of them are equal. This is easy so leaving it.
Just to show that I am capable, we'll prove $\phi, \left\{ \phi \right\} $ are not equal. By definition of $\phi$, there does not exists an  $x \in \phi$ but  $\phi \in \left\{ \phi \right\} $. Hence, not equal.  
\subsection*{3.1.3}


\subsection*{3.1.11}
Show that axiom of replacement implies axiom of specification. Given a set $S$ and a property  $P(x,y)$ pertaining to  $x,y$ such that for every  $x$, there exists at least one $y$ such that $P(x,y)$ is true, then there exists a set $\left\{ y: P(x,y) for some x \in S \right\} $. Axiom of specification is about separating the set into two depending on a property $P$ and which $x \in S$ holds true for $P$ and which ones don't. Let us define a set with those $y$ such that $P(x, y)$ as  that $P(x)$ is true. Therefore  $y \in S$ only. Hence, we show that axiom of specification is implied from axiom of replacement.  $\qed$





\subsection*{Definition 3.6.1 Equal cardinality} Two sets $X$ and  $Y$ are said to have  \textit{equal cardinality} iff there exists a bijection between  $X$ and  $Y$.

Note, we are moving to the next chapter because there is not much to gain from this chapter apart from exercising how to rigorously write mathematics. I know this is the motive of this book but we need to be pragmatic and reach the heart of this book that starts from real numbers. 

\chapter{Integers and rationals}
\section*{Definition and axiom count}

\section{Integers}
We need to define subtraction and we need to define integers. subtraction is an operation and integers is an entity. From the natural numbers construction, we saw, we can define entities by the properties or axioms they hold. Therefore, we will define integers as simply a notation $a--b$. Later on, we'll see that we can just replace $--$ with  $-$.

\subsection{Definition: Integers}
An integer is an expression of the form  $a--b$, where $a,b$ are natural numbers. Two integers $a--b$ and $c--d$ are only equal iff $a+d=b+c$. Note, we are expressing the notion of subtraction using the notion of addition only because the sign $--$ is just a symbol at this point of time without any meaning of an operator as such. We took four natural numbers and defined a relation between them.  

\subsection{Addition and multiplication of integers}
The sum of two integers $a--b$ and  $c--d$ is defined as  $(a--b) + (c--d) := (a+b) -- (c+d)$ 

The product of two integers $a--b$ and $c--d$ is defined as $(a--b) * (c--d) := ac -- bd$. Note, how this is similar to as if the integer is composed of two parts and when two integers interact via an operation, only the parts that are of the same type interact.


\subsection{Addition and Multiplication are well defined for integers}
We need to show if $a,b,a_1,b_1, c,d$ are natural numbers and $a--b=a_1--b_1$, then, $(a--b) + (c--d) = (a_1--b_1) + (c--d)$ and $(a--b) * (c--d) = (a_1-b_1) * (c--d)$ and we also need to show the symmetric-ness of these two equations.

integer as a pair of natural numbers
integer has a negation
subtraction is addition with the negation of integer
a - b = a + (-b) = (a--0) + (0--b) = a -- b
7 - 3 = a + (-3) = (a--0) + (0--3) = 7 -- 3 but here still we haven't reached the stage
of taking away or basically we haven't been able to represent 7 -- 3 rather than what it is already written as.
\subsection*{Remark on how subtraction is conceived from addition}
Note, there is no subtraction in this world defined by Tao. Here's how we get the same quality though. First, we have the concept of negation of an integer. Note, we have defined an integer as a pair of natural numbers written as $a--b$. We define the negation of an integer as  $-(a--b) = (b--a)$. We define subtraction as addition of the negation of the integer, i.e.,  $a-b = a + (-b) = (a--0) + (0--b)$. But, this doesn't tell us how to actually do the reduction of value that we think of. Say,  $7-3 = (7--0)+(0--3)$. Sure, we have expanded this by how do actually get  $4$?  Note, two integers  $a--b$ and  $c--d$ are equal iff $a+d=b+c$. Therefore,  $7--3 = 4--0 $ because  $7+0=3+4$. 

\section{Rationals}
\subsection*{Rationals are not well ordered}
Well ordered means that you can pinpoint the smallest element in a set. You cannot do that with a set of rational numbers. Consider $\left\{ n: 0 < n < 1, n \in \mathbb{Q} \right\} $. For any element you choose, you can find an even smaller element. There is an infinite descent when it comes to rational numbers.

\subsection*{Rationals don't contain all numbers}
We complete this chapter by showing that rationals cannot represent all numbers. There does not exist a number $x$ such that  $x^2 = 2$. If we are able to do this proof and that why such numbers mentioned above don't exist, we have gotten the most out of this chapter. \todo{Prove principle of infinite descent}




\section*{Exercises}
\subsection*{4.1.1} Show that the definition of integers is both reflexive and symmetric. Reflexivity is a property of a relation $R$, i.e., if  $xRx$, then the property is reflexive. Here, the relation $R$ is  $a--b = c--b \iff a+d = b+c$. Then, does this relation hold when both  $a-b$ and  $c-d$ is the same. Let this new integer be  $l--m$ that represents both of them. Then, by the definition of  $R$,  $l--m = l--m \iff l+m = m +l$. Clearly, the definition holds as  $l+m=m+l$ by the symmetric property of addition. 

Symmetric means that $jRk \iff kRj$. Here, $R$ holds  iff $a+d=c+b$. Now, $jRk$ here is  $a+d=c+b$ and clearly  this can be written as  $d+a=b+c$ which is the defintion of  $b--a$. Hence,  $jRk \implies kRj$ and similarly we can show  $kRj \implies jRk$. $\qed$. 



\chapter{Real numbers}
\section{Cauchy sequence}
\subsection*{5.1.8 Definition: Cauchy sequence}
A sequence of rational numbers is said to a cauchy sequence iff for every $\epsilon > 0$, 
the sequence is eventually $\epsilon$-steady. In other words, for every rational number $\epsilon > 0$, there exists a  $N \ge 0$ such that  $d(a_i, a_j) \le \epsilon$ for every  $i, j \ge N$.

\subsection*{5.1.11 Proposition} The sequence $a_n := 1/n$ is a cauchy sequence. 
 \paragraph{Commentary}: We need to show that for any given $\epsilon > 0$, there exists a natural number $N$ such that $|\frac{1}{a_i} - \frac{1}{a_j} |$ \le \epsilon \forall i,j \ge $N$.

\paragraph{Proof: }
Assume we have an arbitrary $\epsilon > 0$. We need to find an appropriate  $N$ such that  $\forall i,j \ge N$,  $|\frac{1}{a_i}-\frac{1}{a_j}| \le \epsilon$. Note, $i \ge N \therefore \frac{1}{i} > 0$. (why (1)?). Similarly, $j \ge N \therefore \frac{1}{j} \le \frac{1}{N}$. (why 2?). $\therefore |\frac{1}{i}- \frac{1}{j}| \le \frac{1}{N}$. Now, it should be enough to show that $\frac{1}{N} \le \epsilon$ or $N > \frac{1}{\epsilon}$. We know that for every ration, there exists a natural number bigger than it by proposition $4.4.1$. $\qed$


Answer to why (1): $i \ge N \implies \frac{1}{i} \ge 0$
 Assume $\frac{1}{i} < 0$. Therefore, by defintion of $<$, there must exists a positive rational  $m$ such that  $\frac{1}{i} + \frac{l}{m} = 0$. By the defintion of addition in fractions, $\frac{m+il}{im} = \frac{0}{1}$. By the defintion of equality of rationals, $m+il = 0*im \implies m + il = 0$.  $l, m \ne 0 $ as it's a positive rational number. Therefore, we have a case where two positive rationals sum up to  $0$ and that is not possible. Hence, a contradiction. Therefore,  $\frac{1}{i} \ge 0$. Why 2 will also have a similar reasoning.

Note, how difficult it is to show that a sequence is a cauchy sequence. We need to explicitly find a $N$ and then show that  $\forall i, j \ge N$, their absolute difference is  $\le \epsilon \forall \epsilon > 0$.

\subsection*{Lemma: 5.1.15 Cauchy sequences are bounded}
Every cauchy sequence is bounded. 
\paragraph{Commentary: } Bounded means that the absolute value of every element in the sequnece is bounded by some rational $M$. Cauchy sequence means that the difference between the two consecutive elements in the sequence gets smaller and smaller.

\paragraph{Proof: }Since, we have a cauchy sequence, we know we will have a number $n$ after which the sequence is  $1$-steady. So, we have two sequences now. One before  $1$-steady and one after $1$-steady. First one is bounded because it is finite. But, why would the second  $1$-steady sequence be bounded? $1$-steady means that the difference between any two elements in the sequence is less than or equal to $1$.  An infinite sequence is bounded if there exists $M$ such that  $a_i \le M \forall n \ge 0$. Clearly, because any consecutive $d(a_i, a_j) \le 1$,
$|a_i - a_j| \le 1$. If $a_i - a_j$ is positive, we have  $a_j \le 1 - a_i$. If negative, we have  $-(a_i - a_j) \le 1 \implies a_j - a_i \le 1 \implies a_j \le 1 + a_i$. In both cases, we found an upper bound for the element. Hence, the second part is also bounded. 
\qed











































































\chapter*{Appendix}
\section{Statements}
\paragraph{Exercises}
\subsection{}
\[
	(x \lor y ) \land \neg (x \land y)
\] 
The negation of the above statement is
\[
\lnot ((x \lor y) \land \neg (x \land y))
.\] 
\[
\lnot (x \lor y) \lor  (x \land y)
.\]
\[
(\neg x \land \neg y) \lor (x \land y)
.\]

\subsection{}
\subsection{}
\paragraph Given: $x \implies y \land \neg x \implies \neg y$
\paragraph{To show: }  $ x \iff y$
\paragraph{Proof: } We need to show $ y \implies x$. 
%Assume $y$ is true. If $x$ is true, we are good but if $x$ is false, $y$ is false (because of $\neg x \implies \neg y $). Hence, if $y$ is true, $x$ must be true as well. $\qed$

\subsection{}
\paragraph{Given: } $x \implies y \land \neg y \implies \neg x$
\paragraph{To show: }  $x \iff y$
\paragraph{Proof:} If $y$ is true, we cannot conclude anything about $x$ hence no we have not shown $x\iff y$.\todo{complete this logic exercise as current proof is wrong} 







\end{document}


