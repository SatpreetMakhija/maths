\documentclass{report}
\usepackage{amsthm, amssymb, amsmath}
\usepackage{todonotes}
\usepackage[colorlinks=true, linkcolor=blue]{hyperref} % Enable hyperlinks

\begin{document}
\listoftodos

\chapter{Introduction}
\section{What is analysis?}
Real analysis is the analysis of real numbers, sequences and series of real numbers and real-valued functions (functions which have range as real numbers). 
\section{Examples}
The examples in this section will show why understanding of real numbers is important. We'll see situations where if we don't really understand what these real numbers are, we won't be to make correct decisions. 
\subsection{division by zero}
$ac = bc \implies a = b$. But, this does not work when $c= 0$. What is this example telling us? Whenever we are cancelling like how we did above, we are ruling out that $c =0 $ for all practical purposes.Because cancellation here actually means division by $c$, we must make sure $c \neq 0$. 
\subsection{divergent series}
Take $S = 1 + \frac{1}{2} + \frac{1}{4} + + \frac{1}{8} + \dots $, what is $S$?. We can use this trick: Multiply both sides by $2$. We get $2S = 2 + 1 + \frac{1}{2} + \frac{1}{4} + \dots$. 
\[
2S = 2 + S \implies S = 2 
.\] 
If we have another sum $S = 1 + 2 + 4 + 8 + 16 + \dots$, using the same trick leads to  \[
2S = 2 + 4 + 8 + 16 + \dots\implies 2S + 1 = S \implies S = -1
.\] 
Clearly, that should not be the case. \todo{EOB need answer} My initial guess is that we cannot apply this trick for a divergent series but can for a convergent series. Why, I need to find out.


\subsection{Divergent sequence}
Let $x$ be a real number and let $L$ define the limit as follows:
\[
L = \lim_{n\to \infty} x^{n} 
.\] 
Let $n = m + 1$. 
\[
\therefore L = \lim_{m+1 \to \infty} x^{m+1}  
.\] 
\[
L = x \lim_{m+1 \to \infty} x^{m}
.\]
\todo{Is it allowed to do take $x$ out of the limit?}
\[
\because m+1 \implies \infty, m \implies \infty 
.\] 
\[
L = x \lim_{m \to \infty} x^{m}
.\] 
\[
L = xL 
.\] 
Either $x = 1$ or $L = 0$. 
This means $L = \lim_{n \to \infty} x^{n} = 0, x \neq 1$. But, this does not make sense because clearly when instantiated for $x = 2$, $L = \lim_{n \to \infty} 2^{n} \neq 0$ 
My guess is that the part where we move $x$ out of the limit so easily should not be allowed.


\subsection{}
\subsection{}
\subsection{Interchanging of integrals}
We sometimes use this trick when integrating $\int \int f(x,y) dx dy = \int \int f(x,y) dy dx$. But this too can lead to issues sometimes. Look at the example in the book for specific example. \todo{EOB need answer on interchanging integrals}

\[
	\int_{0}^{\infty} \int_{0}^{1} \left( e^{-xy}-xye^{-xy}dy dx \right) = \int_{0}^{1} \int_{0}^{\infty} \left( e^{-xy}-xye^{-xy}dxdy \right)      
.\] 
We'll see in the book that the results of the two are different (LHS and RHS is not the same).

\subsection{Interchanging limits}
\subsection{}
\subsection{}


\chapter{Natural numbers}
Just to reiterate the different kinds of numbers we'll deal with in this book, here are they in increasing order of sophistication. Natural numbers, $\mathbb{N}$,  integers, $\mathbb{Z}$, rational numbers, $\mathbb{Q}$, real numbers, $\mathbb{R}$ and complex numbers, $\mathbb{C}$. 

\section{The Peano axioms}
The Peano axioms is one of the ways of defining natural numbers. There are other approaches such as using sets.
We define natural numbers as that set of elements for which the following axioms hold:
\subsection{0 is a natural number.}
\subsection{If $n$ is a natural number, then $n++$ is also a natural number}
Note, we assume that there exists an operation with symbol $++$ that means increment. This axiom allows us to move forward in our count.
\subsection{$0$ is not the successor for any natural number, i.e., $n++ \neq 0 \forall n$}
This ensures we don't circle back to $0$. Imagine if we defined  $\left( 0++ \right) ++ $ as $0$. Axioms $2.1.2$ and $2.1.1$ would still hold but we know we don't want that in our natural number system. This axiom ensures that. 
\subsection{$n \neq m \implies n++ \neq m++$}
This axiom ensures that we don't reach an upper bound. If we define $2++$ and  $3++$ both as $3$, none of the above axioms are violated and yet we know we don't want an upper bound to our natural number system. Take the next number, $(3++)++ = 3++ = 3$, we have reached an upper bound for us.
Another way to state this axiom is in its contrapositive form. $n++ = m++ \implies n = m$.
\subsection{Principle of Mathematical Induction}
It might be surprising right now why we are including this as an axiom because it is part of the larger logic scheme. Why do we need to explicitly say this. My guess is that it is needed because there are other systems where this principle does not hold. \todo{find why math induction principle is needed as a Peano axiom to define natural numbers}
The principle is if some property $P$ holds for $0$, that is, if $\left( P\left( 0 \right)  \right) $ is true and  $P\left( n \right) \implies P\left( n+1 \right) $ is true, then property $P$ is true for all elements in the number system. 

This axiom allows us to define properties of objects for all elements.

\paragraph{Note our definition of natural numbers is axiomatic and not \textit{constructive}.} What I mean by that is we have not defined \textit{what} a natural number is but rather properties that a natural number holds. Any number system that follows the above mentioned axioms is a natural number system.

\subsection{Recursive definition of sequence: Proposition} Suppose for each natural number $n$, we have some function $f_n: \mathbb{N} \mapsto \mathbb{N}$. Let $c$ be a natural number. Then, we can assign a unique natural number $a_n$ to each natural number $n$ such that $a_0 = c$ and $a_\left\{ n++ \right\} = f_n\left( a_n \right) $ for each natural number $n$.
\paragraph{commentary} The goal is to show that each value of the sequence $a_i$ is defined only once and because we are looking at a recursive defintion we do not revisit $a_i$ again when going forward. 
\paragraph{proof} We'll prove this by induction. Take the base case, $a_0 = c$. Because of axiom $2.1.3$, going forward, any $a_{m++} $ will not redefine $a_0$. For example, if axiom $2.1.3$ was actually violated and say $3++ = 0$. Then, $a_{3++} = f_3\left( a_3 \right) = a_0$. Note, we redefined $a_0$ without this axiom.

Let's assume that $a_n$ by the given recursive formula uniquely defined $a_n$. We need to this also holds for $a_\left( n+1 \right) $. Any natural number after $n++$, say we call it $m++ $ will redefine $n++$ because of axiom $2.1.4$. Note, this axiom was introduced to ensure the successors are always bigger and that's what this axiom ensures. Because we cannot circle back, the previously defined $a_i$ remains as is.

\section{Addition}
\subsection{Definition: Addition of natural numbers}
Let $m$ be a natural number, we define addition of zero to $m$ as $0 + m := m$. We give a recursive defintion for adding. Let's say we know how to perform $n + m$. Therefore, adding  $n++$ to $m$ is defined as $\left( n++ \right) +m := \left( n+m \right) ++$.

For example, say, we need to add $2 + 5$.  $\left( 1++ \right) +5 = \left( 1 + 5 \right) ++ = \left( \left( 0++ \right) +5 \right) ++ = \left( \left( 0 + 5 \right) ++ \right) ++ = \left( 5++ \right) ++ = 6++ = 7 $
\paragraph{Commentary:} This definition is enough to deduce everything about addition of natural numbers!
\subsection{Lemma} For any natural number $n$, $n+0 = n$. 


\paragraph{Commentary:} Note, in our recursive defintion, we have defined only $0+n=n$ and we don't know yet that for any $a,b \in \mathbb{N}$, $a+b=b+a$.
\paragraph{Proof:} Take base case, $n = 0$. $0+0=0$. This is true by our recursive defition's base case. Assume, $n+0 = n$ for $n$. We need to show $(n++)+0 = n++$ and we are done with our proof by induction. 

 $\left( n++ \right) +0 = \left( n + 0 \right) ++ $ by our defintion of addition. $\left( n+0 \right) ++ = n++$ because of our induction assumption for $n$. Hence, proved.

 \subsection{Lemma} For any natural numbers $n,m$ $n+\left( m++ \right) = \left( n+m \right) ++$.

\paragraph{Commentary:}Again, we cannot deduce this from our recursive defintion  $\left( n++ \right) + m = \left( n+m \right) ++$ because we do not know $a+b = b+a$. 

\paragraph{Proof: } We'll try this by induction on $n$ (and keep $m$ fixed). Let's take the base case at $n=0$. $0+\left( m++ \right) = m++ = \left( 0 + m \right) ++ $. The stamenent holds for the base case.

Assume, $n+\left( m++ \right) = \left( n+m \right) ++$. We now need to show that $\left( n++ \right) + \left( m++ \right) = \left( \left( n++ \right) + m \right) ++ $

$\left( n++ \right) + \left( m++ \right) = \left( n+\left( m++ \right)  \right) ++ $ by the recursive defintion of addition.
$\left( n + \left( m++ \right)  \right) ++ = \left( \left( n+m \right) ++ \right) ++ $ by the the induction step taken for $n$. 
$\left( \left( n++ \right) +m \right) ++$ by the defintion of recursive defntion of addition. Hence, proved. 

\subsection*{Corollary}
$n++ = n + 1 $ 
\paragraph{Commentary: }Why would this hold because of Lemma  $2.2.2$ and Lemma $2.2.3$. 

\paragraph{Proof:} $n+1 = n + \left( 0++ \right) = \left( n+0 \right) ++$ by Lemma $2.2.3$. $\left( n+0 \right) ++ = n++$ because of Lemma $2.2.2$. Hence, proved.

\subsection{Proposition} Addition is commutative. For any natural numbers $n,m$, $n+m = m+n$. 

\paragraph{Proof: } We can prove this by induction. (note, anytime there's a proof of applying it for all numbers, induction really helps). We'll fix $m$. Take  $n=0$. $0+m = m$ by defition of base case for addition. $m+0 = m$ by Lemma $2.2.2$. Base case is done.

Assume $n+m = m+n$. This is our induction step. We now need to show it holds for $n++$, i.e., $\left( n++ \right) + m = m + \left( n++ \right) $. 
LHS, $\left( n++ \right) + m = \left( n+m \right) ++	$ by the recursive definition of addition. $\left( n+m \right) ++ = \left( m+n \right) ++$ by our induction step.
RHS, $m+ \left( n++ \right) = \left( m+n \right)++ $ by Lemma $2.2.3$.

LHS = RHS. Hence proved. $\qed$

\subsection{Proposition} Addition is associative. For any natural numbers $a,b,c$, we have $(a+b)+c = a+\left( b+c \right) $.
\todo{Write proof for Addition is associative}

\subsection{Proposition} Cancellation law. Let  $a,b,c$ be natural numbers such that $a+b = a + c$. Then we have $b=c$.
 \paragraph{Commentary: }Note, we do not have the concept of negative numbers or subtraction yet.
\paragraph{Proof: }We'll prove this by induction. Fix $b,c$. Take base case,  $a = 0 $. $0+b = 0 + c \implies b = c$ because of base case of definition of addition.

Assume $a+b = a+c \implies b = c$. We need to show $\left( a++ \right) +b =  \left( a ++ \right) + c \implies b = c $.

\[
\left( a++ \right) + b = \left( a++ \right) +c 
.\] 
 \[
	 \left( a+b \right) ++ = \left( a+c \right) ++ \tag{by defintion of addition}
.\]
\[
	b++ = c++ \tag{by assumption of the induction step}
.\] 
\[
b = c
.\] 
Hence, proved by induction.  $\qed$

\subsection{Definition} Positive natural numbers. A natural number $n$ is said to be positive iff it is not equal to $0$. 

\subsection{Proposition} If $a$ is positive and $b$ is a natural number, then $a+b$ is positive.
 \paragraph{Proof: } We'll use induction on $b$. Let $b = 0$. $a + b = a + 0 =  a$. Base case holds. Assume $a+b$ is positive. Then we need to show $a+\left( b++ \right) $ is positive. $a + \left( b++ \right) = \left( a+b \right) ++$ by Lemma 2.2.3. $\left( a+b \right) ++ $ cannot be zero because of Axiom $2.3$ ($n++ \neq 0$ ). Hence, $\left( a+b \right) ++$ is positive. This closes the induction. $\qed$

 \subsection{Corollary} If $a$ and $b$ are natural numbers such that $a + b = 0$ then $a=b=0$.
\paragraph{Proof: } We'll prove this by contradiction. Assume $a+b = 0 $ but $a \neq 0$ or $b \neq 0$. In either case, $a+b=0$ where one of them is not zero. But, by Proposition $2.2.8$ that cannot be the case as $a+b$ must be positive (not zero). Hence, contradiction and $a = b= 0$.  $\qed$


\subsection{Lemma} Let  $a$ be a positive number. Then there exists exactly one natural number $b$ such that $b++ = a$.
\paragraph{Commentary: } Note, this means there is exactly one element behind $a$. There seems to be an order between the natural numbers.
\paragraph{Proof}: \todo{Prove}

\subsection{Definition} Ordering of natural numbers. Let $n$ and $m$ be natural numbers. We say $n$ is \italics{greater than or equal to $m$} iff we have  $n = m + a$ for some natural number $a$. We write \italics{greater than or equal to} as $n\geq m $ or  $m \leq n$. We say that $n$ is strictly greater than $m$ iff $n \geq m$ and $n \neq m$. 

\paragraph{Commentary: }Note, how we have used the concept of some $a$ existing such that adding it to $m$ gives us $n$. This also helps us show why there is no largest natural number because for every $n$, $n++ > n$. Hence, for the number $n ++$, there's another larger number $\left( n++ \right) ++$. 


\subsection{Proposition} Basic properties of order for natural numbers. Let $a,b,c \in \mathbb{N}$. Then
\begin{enumerate}
	\item (Order is reflexive) $a \geq a$
	\item (Order is transitive) If  $a \geq b$ and $b \geq c$, then $a \geq c$
	\item (Order is anti-symmetric) If  $a \geq b$ and $b \geq$, then $a =b$.
	\item (Addition preserves order)  $a \geq b$ iff $a+c \geq b+c$
	\item  $a < b$ iff $a++ \leq b$.
	\item  $a < b $ iff $b = a + d$ for some positive number $d$. 
\end{enumerate}
\paragraph{Proof: }
\begin{enumerate}
	\item 1
		$a \ge a$ iff there exists  $n$ such that $a = a + n$. Let $n=0$. Therefore, $a= a + 0 = a$. Hence, proved.
	\item
		It is given that  $a \ge b$ and $b \ge c$. We need to show this implies $a \ge c$. There must bexist $m$ such that  $a = b + m$ and  $n$ such that $b = c + n$. Substituting value of $b$, $a = c + n + m \implies a = k + c$ where  $k$ is some natural number. Hence, by definition of ordering of natural numbers, $a \ge c$.  $\qed$   
	\item
		 $a = b + m$ for some  $m$ and $b = a + k$ for some $k$. Substitute value of $b$, $a = a + k + m \implies a + 0 = a + (k+m) \implies 0 = k +m $ by cancellation law (Proposition $2.2.6$). The only way this holds is iff $k = m = 0$.
	This is because there are four situations $k,m \ne 0 \lor k=m=0, \lor k = 0 \land m \ne 0 \lor k \ne 0 \land m=0$. In all situations except $k=m=0$, we'll have a situation where  Axiom  $2.1.3$ is violated because we'll have an equation of form  $n++=0$. Hence, $a = b + m = 0 = b \implies a = b$. $\qed$ 
\item 
	Assume $a \ge b$. This means $a = b + m$ for some $m$. Adding $c$ on both sides,   (for which we'll need to prove this logic TODO), $a + c = b + m + c \implies \left( a +c \right) = m + (b+c)$. (We have only rearranged the terms using Proposition $2.2.4$). Therefore, by definition of greater than, we have  $a + c \ge b+c$.
	Now assume  $a +c \ge b + c $. We need to show  $a \ge b$.  $a +c= b +c + d$ for some  $d$. 
By cancellation law, $a = b + d$. Hence, $a \ge b$.  We are done with our proof now.

\item 
Assume 
\end{enumerate}














































































\chapter*{Appendix}
\section{Statements}
\paragraph{Exercises}
\subsection{}
\[
	(x \lor y ) \land \neg (x \land y)
\] 
The negation of the above statement is
\[
\lnot ((x \lor y) \land \neg (x \land y))
.\] 
\[
\lnot (x \lor y) \lor  (x \land y)
.\]
\[
(\neg x \land \neg y) \lor (x \land y)
.\]

\subsection{}
\subsection{}
\paragraph Given: $x \implies y \land \neg x \implies \neg y$
\paragraph{To show: }  $ x \iff y$
\paragraph{Proof: } We need to show $ y \implies x$. 
%Assume $y$ is true. If $x$ is true, we are good but if $x$ is false, $y$ is false (because of $\neg x \implies \neg y $). Hence, if $y$ is true, $x$ must be true as well. $\qed$

\subsection{}
\paragraph{Given: } $x \implies y \land \neg y \implies \neg x$
\paragraph{To show: }  $x \iff y$
\paragraph{Proof:} If $y$ is true, we cannot conclude anything about $x$ hence no we have not shown $x\iff y$.\todo{complete this logic exercise as current proof is wrong} 







\end{document}


